\documentclass[aspectratio=169]{beamer}

\title[Short Title]{
    ECE 555 Group Presentation
}
\author{
	% \tiny
	Igor Semyonov
	\and Jordan Carnes
	\and Robert Laverne Griffin
}
\institute{
    George Macon University, Department of Electrical and Computer Engineering
}

\usepackage[T1]{fontenc}
\usepackage{authblk}
\usepackage{float}
\usepackage{caption}
\usepackage{subcaption}
\usepackage[letterpaper, margin=1in]{geometry}
\usepackage[english]{babel}
\usepackage{hyperref}
\usepackage{booktabs}
\usepackage{tabularx}
\usepackage{makecell}
\usepackage{multirow}

\usepackage{fancyhdr}

\usepackage{amsmath,amsfonts,amssymb,mathrsfs,mathtools}
\DeclarePairedDelimiter\abs{\lvert}{\rvert}%
\DeclarePairedDelimiter\norm{\lVert}{\rVert}%
% Swap the definition of \abs* and \norm*, so that \abs
% and \norm resizes the size of the brackets, and the 
% starred version does not.
\makeatletter
\let\oldabs\abs
\def\abs{\@ifstar{\oldabs}{\oldabs*}}
%
\let\oldnorm\norm
\def\norm{\@ifstar{\oldnorm}{\oldnorm*}}
\makeatother
%%%%%%%%%%%%%%%%%%%

% references for equations
\makeatletter
\def\tagform@#1{\maketag@@@{\bfseries(\ignorespaces#1\unskip\@@italiccorr)}}
\renewcommand{\eqref}[1]{Eq. \textup{{\normalfont(\ref{#1}}\normalfont)}}
\makeatother
\newcommand{\figref}[1]{Figure \ref{#1}}
\newcommand{\lstref}[1]{Listing \ref{#1}}
\newcommand{\tabref}[1]{Table \ref{#1}}

\usepackage{listings, listings-rust}
\usepackage[framemethod=TikZ]{mdframed}
\lstset
{
	language=C,
	basicstyle=\footnotesize,
	numbers=left,
	numberstyle=\tiny,
	stepnumber=1,
	showstringspaces=false,
	tabsize=1,
	breaklines=true,
	breakatwhitespace=false,
	xleftmargin=-8pt,
	frame=T,
}
\newcommand{\codefileNoFigure}[3]{
	\mdframed[roundcorner=5pt, backgroundcolor=blue!30]
	% \lstinputlisting[caption=\texttt{#1} {#2}, label=#3]{#1}
	\lstinputlisting[caption={#2}, label=#3]{#1}
	\endmdframed
}
\newcommand{\codefileRustNoFigure}[3]{
	\mdframed[roundcorner=5pt, backgroundcolor=blue!30]
	\lstinputlisting[
		language=Rust,
		caption={#2},
		label=#3
	]{#1}
	\endmdframed
}
\newcommand{\codefile}[3]{
	\begin{figure}[H]
		\codefileNoFigure{#1}{#2}{#3}
	\end{figure}
}

\usepackage{enumitem}
\newlist{problems}{enumerate}{2}
\setlist[problems]{
	label={Problem \arabic*.},
	ref={\arabic*},
	wide,
	itemindent=0pt,
	listparindent=0pt,
}
\newlist{subproblems}{enumerate}{2}
\setlist[subproblems]{
	label={(\alph*)},
	ref=\alph*
	wide,
	listparindent=0pt,
	itemindent=0pt,
}
\usepackage{scalerel}
\setcounter{MaxMatrixCols}{20}
\newcommand{\longdiv}{\smash{\mkern-0.43mu\vstretch{1.31}{\hstretch{.7}{)}}\mkern-5.2mu\vstretch{1.31}{\hstretch{.7}{)}}}}

% \usepackage{xparse}

\usepackage{tabto}
\NumTabs{6}

\newcommand{\eq}[2]{
    \begin{equation}\label{#1}
        #2
    \end{equation}
}

\usepackage{xspace}
\newcommand{\latex}{\LaTeX\xspace}
\newcommand{\tex}{\TeX\xspace}

\newcommand{\R}{\mathbb{R}}
\newcommand{\Rn}[1][n]{\mathbb{R}^#1}
\NewDocumentCommand{\Rnp}{ O{n} O{p} }{\mathbb{R}^#1_#2}


\def \Hinv {H_\text{inv}}
\def \hinv {h_\text{inv}}


\begin{document}

\begin{frame}
    \vspace{-1.8cm}
	\titlepage
\end{frame}

\begin{frame}
	\frametitle{Why not C?}
    \framesubtitle{}

    It should not be used for production.

    See \url{https://veresov.pro/cmustdie/}
\end{frame}

\begin{frame}
	\frametitle{Why Rust?}
    \framesubtitle{Type System}
\end{frame}

\begin{frame}
	\frametitle{Why Rust?}
    \framesubtitle{Safety}
\end{frame}

\begin{frame}
	\frametitle{Why Rust?}
    \framesubtitle{Ergonomics while remaining fast}

    Here I may include my rust implementation of project 1 and compare it to my C version in both ergonomics, safety, and speed.
\end{frame}

\begin{frame}
	\frametitle{Rust and CUDA}
    \framesubtitle{Problems regarding shared memory in Rust}
\end{frame}

\begin{frame}
	\frametitle{Rust and CUDA}
    \framesubtitle{Current options}

    \begin{outline}[itemize]
        \1 Rust GPU
        \1 Rust CUDA
    \end{outline}
\end{frame}

\begin{frame}
	\frametitle{Rust and CUDA}
    \framesubtitle{Focusing on Rust CUDA}

    Currently being rebooted and is in active development.

    Uses nvidia's nvvm tool which is built on LLVM 7.
\end{frame}

\begin{frame}
	\frametitle{What is a compiler?}
    \framesubtitle{}

    Description of problems when going straing from source code to machine code.
\end{frame}

\begin{frame}
	\frametitle{The problem LLVM solves}
    \framesubtitle{}

    Description of LLVM and it's intermediate representation and how this has enabled much easier language development.
\end{frame}

\begin{frame}
	\frametitle{NVVM}
    \framesubtitle{}

    How NVVM works
\end{frame}

\begin{frame}
	\frametitle{Rust CUDA and NVVM}
    \framesubtitle{}

    How NVVM is used in Rust CUDA
\end{frame}

\begin{frame}
	\frametitle{This is a slide}
    \framesubtitle{With a subtitle}

	\begin{columns}
		\begin{column}{0.6\textwidth}
            This is some text in a column. Could be a figure instead.
		\end{column}
		\begin{column}{0.4\textwidth}
			\begin{outline}[itemize]
				\1
                This is a list
				\1
                It is an itemized one
				\1
                Hence the bullets
			\end{outline}
		\end{column}
	\end{columns}
\end{frame}

% \begin{frame}%[t, allowframebreaks]
%     \frametitle{
%         References
%     }
%     % \nocite{thing-we-want-listed-even-though-it-was-never-cited}
%     % \bibliographystyle{amsalpha}
%     \bibliographystyle{apalike}
%     % If you have more than one page of references, you want to tell beamer
%     % to put the continuation section label from the second slide onwards
%     \setbeamertemplate{frametitle continuation}[from second]
%     % and kill the abominable icon
%     \setbeamertemplate{bibliography item}{\Huge$\rightharpoonup$}
%     % \bibliographystyle{siam}
% \end{frame}
% \bibliographystyle{amsalpha}
% \bibliography{../refs}

\end{document}
